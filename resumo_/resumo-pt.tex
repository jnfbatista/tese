%-----------------------------------------------
% Template para criação de resumos de projectos/dissertação
% jlopes AT fe.up.pt,   Fri Jul  3 11:08:59 2009
%-----------------------------------------------

\documentclass[9pt,a4paper]{extarticle}

%% English version: comment first, uncomment second
\usepackage[portuguese]{babel}  % Portuguese
%\usepackage[english]{babel}     % English
\usepackage{graphicx}           % images .png or .pdf w/ pdflatex OR .eps w/ latex
\usepackage{times}              % use Times type-1 fonts
\usepackage[utf8]{inputenc}     % 8 bits using UTF-8
\usepackage{url}                % URLs
\usepackage{multicol}           % twocolumn, etc
\usepackage{float}              % improve figures & tables floating
\usepackage[tableposition=top]{caption} % captions
%% English version: comment first (maybe)
\usepackage{indentfirst}        % portuguese standard for paragraphs
%\usepackage{parskip}

%% page layout
\usepackage[a4paper,margin=30mm,noheadfoot]{geometry}

%% space between columns
\columnsep 12mm

%% headers & footers
\pagestyle{empty}

%% figure & table caption
\captionsetup{figurename=Fig.,tablename=Tab.,labelsep=endash,font=bf,skip=.5\baselineskip}

%% heading
\makeatletter
\renewcommand*{\@seccntformat}[1]{%
  \csname the#1\endcsname.\quad
}
\makeatother

%% avoid widows and orphans
\clubpenalty=300
\widowpenalty=300

\begin{document}

\title{\vspace*{-8mm}\textbf{\textsc{Sistema de Reconhecimento de Objectos\\para Demonstrador de Condução Robótica Autónoma}}}
\author{\emph{João Nuno Ferreira Batista}\\[2mm]
\small{Dissertação realizado sob a orientação do \emph{Prof.\ Armando Jorge Sousa}}}\\
\date{}
\maketitle
%no page number 
\thispagestyle{empty}

\vspace*{-4mm}\noindent\rule{\textwidth}{0.4pt}\vspace*{4mm}



\section{Resumo}\label{sec:motiva}

O aparecimento de dispositivos RGBD, ou seja, sensores que além de captarem imagens RGB,
(como qualquer câmara) também captam informação de profundidade, disponíveis para o utilizador
comum mudaram a pois tornou acessível um sensor que de outra forma seria demasiado caro
para ser considerado a qualquer investigador. Desta forma, pode-se equipar um robô de demonstrações de
robótica autónoma com um \emph{Kinect} e usufruir de um sensor de profundidade.

Fazendo o robô reagir a objectos simples, tal como perseguir uma esfera ou
fugir de um cilindro e mesmo a combinação de ambos os comportamentos, permite 
criar demonstrações interactivas e apelativas para sensibilizar a assistência 
ao mundo da robótica e do que através dela se pode criar. Para concretizar estes objectivos,
tirando partido da disponibilidade deste tipo de sensor, torna-se necessário desenvolver 
software de reconhecimento em tempo real de objectos simples (esferas, cones
e cilindros) para poderem ser utilizados nas demonstrações de robótica
autónoma. 

Todo este trabalho implica uma pesquisa científica cuidada dos métodos e técnicas
que existem para o reconhecimento dos ditos objectos e também dos de diferentes tipos
de demonstradores que existem que servem diferentes propósitos mas que demonstram
um exemplo de robótica autónoma.


%%English version: comment first, uncomment second
\bibliographystyle{unsrt-pt}  % numeric, unsorted refs
%\bibliographystyle{unsrt}  % numeric, unsorted refs
\bibliography{refs}



\end{document}
