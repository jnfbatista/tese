\chapter{Conclusões e Trabalho Futuro} \label{chap:concl}

\section*{}

Deve ser apresentado um resumo do trabalho realizado e apreciada a
satisfação dos objetivos do trabalho, uma lista de contribuições
principais do trabalho e as direções para trabalho futuro.

A escrita deste capítulo deve ser orientada para a total compreensão
do trabalho, tendo em atenção que, depois de ler o Resumo e a
Introdução, a maioria dos leitores passará à leitura deste capítulo de
conclusões e recomendações para trabalho futuro.

\section{Satisfação dos Objectivos}

 

\section{Trabalho Futuro}

No que diz respeito a trabalho futuro, existem várias melhorias que poderiam ser introduzidas, contudo as mais concretas seriam em dois campos muito particulares: rapidez de reconhecimento e a forma como a avaliação é realizada.

No que diz respeito à rapidez de reconhecimento, tendo em conta que a rapidez de identificação é um requisito muito importante no que diz respeito à robótica autónoma, poder-se-á potenciar a velocidade de reconhecimento usando bibliotecas de paralelização (como a OpenMP) para avaliar os \emph{clusters}, encontrados na imagem, em paralelo. Outra das hipóteses seria usar, por exemplo, algoritmos genéticos para otimizar os fatores envolvidos na deteção de \emph{clusters} pois é a operação mais custosa em termos de tempo no processo global\footnote{fundamentar com dados}.

Quanto à avaliação, 



%\vspace*{12mm}


