\chapter{Introdução} \label{chap:intro}

\section{Enquadramento} \label{sec:context}

Esta dissertação de Mestrado surgiu de um acumular de factores importantes que lhe dão 
contexto e definem o âmbito da mesma.
Um dos factores que levaram a que esta dissertação fosse proposta foram o facto de ter sido
desenvolvido, no âmbito de uma tese de Mestrado do curso de Mestrado Integrado em
Engenharia Electrotécnica e Computadores, um demonstrador de robótica autónoma. 

Entretanto também foi disponibilizado no mercado, um sensor de RGBD da \emph{Microsoft} chamado 
\emph{Kinect} que foi desenvolvido para o sistemas de jogos \emph{X-Box 360} de modo a
se interagir com jogos totalmente sem comandos, respondendo, desta forma, ao movimentos e gestos
dos jogadores. Pouco depois da sua saída para o mercado foi desenvolvido um controlador \emph{opensource}
sendo que ficou aberta a possibilidade de se utilizar o \emph{Kinect} como um sensor em qualquer
robô com portas USB.

Estes dois factores, aliados ao desejo de se fazer demonstrações portáveis de robótica autónoma 
tornaram possível esta dissertação, em que se estuda a identificação de objectos em 3D 
em tempo real recorrendo à informação de profundidade que o \emph{Kinect} fornece.

Uma noção basilar é que o tipo de informação que será processada serão percepções, portanto
será sempre a informação obtida pelo \emph{Kinect} mas com contexto, sendo essa percepção
considerada sempre em unidades SI.

\section{Projecto} \label{sec:proj}

Resultante desta dissertação, existirá um projecto desenvolvido que permite que o robô,
que já se encontra desenvolvido, faça demonstrações de robótica autónoma, usando como sensor
principal o \emph{Kinect}. Este software permitirá configurar vários parâmetros, 
de uma forma intuitiva, para ser possível fazer diferentes demonstrações combinando
configurações diferentes.

\section{Motivação e Objectivos} \label{sec:goals}


A motivação principal desta dissertação é fazer o reconhecimento de objectos em tempo 
real através do sensor de profundidade (utilizando o \emph{Kinect}), e após reconhecidos
pelo sistema, serem utilizados para fazer demonstrações de robótica autónoma. Essas
demonstrações serão possíveis através da programação de comportamentos que serão
sempre relativos a esses objectos.

Esta tese pode ser considerada um sucesso se, além do software do demonstrador, se conseguir
reconhecer vários objectos através do sensor de profundidade, tais como esferas, cones, cilindros
e paralelepípedos.


\section{Estrutura da Dissertação} \label{sec:struct}

Para além da introdução, esta dissertação contém mais um capítulo.
No capítulo~\ref{chap:sota}, é descrito o estado da arte e são
apresentados trabalhos relacionados.
