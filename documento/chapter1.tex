\chapter{Introdução} \label{chap:intro}

\section{Enquadramento} \label{sec:context}

Esta dissertação de Mestrado surgiu de um acumular de fatores importantes que lhe dão contexto e definem o âmbito da mesma, cuja listagem e descrição se seguem.

Um dos fatores que levaram a que esta dissertação fosse proposta foram o facto de ter sido desenvolvido, no âmbito de uma tese de Mestrado do curso de Mestrado Integrado em Engenharia Electrotécnica e Computadores, um demonstrador de robótica autónoma com vista a fazer demonstrações e também participar em competições de condução autónoma. 

Entretanto também foi disponibilizado no mercado, um sensor de RGBD da \emph{Microsoft} chamado \emph{Kinect} que foi desenvolvido para o sistemas de jogos \emph{X-Box 360} de modo a se interagir com jogos e software totalmente sem comandos, respondendo, desta forma, ao movimentos e gestos dos jogadores. Pouco depois da sua saída para o mercado foram desenvolvidos controladores \emph{opensource}, sendo que ficou aberta a possibilidade de se utilizar o \emph{Kinect} como um sensor em qualquer robô com portas USB.

Estes dois fatores, aliados ao desejo de se fazer demonstrações portáveis de robótica autónoma tornaram possível esta dissertação, em que se estuda a identificação de objetos 3D em tempo real recorrendo à informação de profundidade que o \emph{Kinect} fornece.

Uma noção basilar é que o tipo de informação a ser processada serão perceções, portanto será sempre a informação obtida pelo \emph{Kinect} mas com contexto, sendo essa perceção considerada sempre em unidades SI.

Em suma, o que se pretende é, utilizando o \emph{Kinect}, criar um sistema que reconheça objectos para um robô que já existe.

\section{Projecto} \label{sec:proj}

Além deste documento que expõe o conhecimento e as tecnologias utilizadas para resolver o problema, e analisa de forma crítica os resultados obtidos, existe também um projeto desenvolvido que permite o reconhecimento de objetos 3D. Este projeto estará também já preparado para se integrar num sistema de um robô que já se encontra desenvolvido, que usa como sensor principal o \emph{Kinect}. Este software permitirá configurar vários parâmetros, de uma forma intuitiva, para ser possível fazer o reconhecimento de objetos 3D e que seja claro para o utilizador final o que está a acontecer.

\section{Motivação e Objetivos} \label{sec:goals}


A motivação principal desta dissertação é fazer o reconhecimento de objetos em tempo real através do sensor de profundidade (utilizando o \emph{Kinect}), e após reconhecidos pelo sistema, serem utilizados para fazer demonstrações de robótica autónoma. As características mais importantes serão: o reconhecimento de objetos 3D em tempo real e a identificação das características dos mesmos.

Esta tese pode ser considerada um sucesso se for possível  reconhecer vários objetos simples através do sensor de profundidade, tais como planos, cilindros, paralelepípedo e de objetos complexos que sejam compostos pelos objetos simples.

Adicionalmente serve como uma forma de avaliar a capacidade do \emph{Kinect}, como sensor RGBD, de reconhecer objetos do dia a dia e testar as suas limitações.


\section{Estrutura da Dissertação} \label{sec:struct}

Para além da introdução, nesta dissertação é descrito o estado da arte e são
apresentados trabalhos relacionados. É também feita a descrição da implementação e apresentam-se os resultados extraídos desta dissertação.
