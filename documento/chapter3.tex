\chapter{Reconhecimento de objectos}\label{chap:chap3}

\section*{}

Este capítulo deve começar por fazer uma apresentação detalhada do
problema a resolver\footnote{Na introdução a apresentação do
  problema foi breve.} podendo mesmo, caso se justifique,
constituir-se um capítulo com essa finalidade.

Deve depois dedicar-se à apresentação da solução sem detalhes de
implementação. 
Dependendo do trabalho, pode ser uma descrição mais teórica, mais
'arquitectural', etc.

\section{Reconhecimento de Objectos}

% Provavelmente transladar isto para o capítulo 1
Um dos grandes problemas da condução autónoma, e naturalmente dos seus demonstradores, é o
da localização, isto é: como é que um robô autónomo sabe a sua localização no espaço. A 
forma mais simples de resolver este problema é permitir que o robô faça o mapeamento do que
o rodeia de forma a se situar relativamente ao ambiente em que se insere, nomeadamente aos 
objectos que o compõem.
Os objectos que compõem o seu ambiente, sejam eles simples ou devidadamente, complexos
identificados e situados numa representação interna de um mapa permite que o robô interaja de 
forma mais eficaz com o ambiente.

E é precisamente esse problema que esta tese se propõe a resolver, através do reconhecimento de
objectos.
% fim de parte supérflua

Como o objectivo desta tese é o reconhecimento de objectos através das percepções de um 
robô decidiu-se fazer inicialmente o reconhecimento de planos no espaço e de objectos 3D
simples e depois juntar essas capacidades de reconhecimentos para reconhecer objectos 
complexos que constituam uma composição dos mais simples que já é possível detectar.

\subsection{Reconhecimento de Objectos Simples}
Os objectos simples aqui referenciados, não são mais do que as formas elementares existentes
no 3D:
\begin{itemize}
\item Esferas;
\item Cilindros;
\item Toróides;
\item Pirâmides;
\item Cubos.
\end{itemize}

Estas primitivas geométricas (no âmbito do 3D) e o seu rápido reconhecimento permitem que o
demonstrador de robótica autónoma tenha um conhecimento mais profundo do ambiente que o 
rodeia e com o qual irá interagir.

Para o reconhecimento destes objectos 
 
\subsection{Reconhecimento de Objectos Complexos}

\section{ Procedimento Experimental}


\section{Resumo e Conclusões}

Resumir e apresentar as conclusões que se podem tirar no fim deste
capítulo.
