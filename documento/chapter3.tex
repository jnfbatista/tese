\chapter{Reconhecimento de objectos}\label{chap:chap3}

\section*{}

Este capítulo deve começar por fazer uma apresentação detalhada do
problema a resolver\footnote{Na introdução a apresentação do
  problema foi breve.} podendo mesmo, caso se justifique,
constituir-se um capítulo com essa finalidade.

Deve depois dedicar-se à apresentação da solução sem detalhes de
implementação. 
Dependendo do trabalho, pode ser uma descrição mais teórica, mais
'arquitectural', etc.

\section{Reconhecimento de Objectos}



O objectivo desta tese é o reconhecimento de objectos através das percepções de um 
robô, tirando o máximo partido dos sensores 3D que recentemente apareceram no mercado para o
utilizador comum. Utilizando tecnologia já desenvolvida, como os sensores 3D e bibliotecas
informáticas \footnote{????}, será desenvolvida uma aplicação onde se faça o reconhecimento de
estruturas \footnote{ou objectos} simples, nomeadamente planos, paralelipípedos e cilindros, para
se utilizar no reconhecimento de objectos mais complexos dos quais estes fazem parte.

Visto que o objectivo último é a integração no software de um demonstrador de robótica autónoma,
um objectivo secundário é que o reconhecimento seja feito num curto espaço de tempo para maximizar
a performance do demonstrador.

% Provavelmente transladar isto para o capítulo 1
%Um dos grandes problemas da condução autónoma, e naturalmente dos seus demonstradores, é o
%da localização, isto é: como é que um robô autónomo sabe a sua localização no espaço. A 
%forma mais simples de resolver este problema é permitir que o robô faça o mapeamento do que
%o rodeia de forma a se situar relativamente ao ambiente em que se insere, nomeadamente aos 
%objectos que o compõem.
%Os objectos que compõem o seu ambiente, sejam eles simples ou devidadamente, complexos
%identificados e situados numa representação interna de um mapa permite que o robô interaja de 
%forma mais eficaz com o ambiente.
%
%E é precisamente esse problema que esta tese se propõe a resolver, através do reconhecimento de
%objectos.
% fim de parte supérflua

O trabalho desenvolvido também poderá ser utilizado, no demostrador autónomo, para resolver o problema
de localização, isto é: como é que um robô autónomo sabe a sua localização no espaço. Havendo 
reconhecimento de objectos permitirá que o robô utilize a posição dos objectos reconhecidos para 
mais facilmente se localizar no espaço.



\section{ Solução do Problema}

Para resolver o problema do reconhecimento de objectos em 3D decidiu-se seguir uma estratégia \emph{bottom-up}
tratando primeiro do reconheciemtno de objectos simples, nomedamentede planos no espaço, cilindros e paralelipípedos,
e de seguida reconhecer mesas em cuja constituição se encontram esses objectos.

O sensor utilizado para a captura de informação 3D foi o \emph{Kinect} da \emph{Microsoft} que é o mais comum no mercado
e também foi o que a comunidade \emph{opensource} mais rápidamente adoptou, desenvolvendo controladores e bibliotecas
que permitem que o desenvolvimento seja multiplataforma


\subsection{Reconhecimento de Objectos Simples}

No que diz respeito a objectos simples optou-se pelos seguintes que são os constituíntes mais comuns
do principal caso de estudo dos objectos complexos:
\begin{itemize}
\item Cilindros;
\item Paralelipipedos;
\item Planos;
\end{itemize}

O reconhecimento destes objectos mais simples terá associado um valor de confiança para se poder
avaliar a qualidade do método.\footnote{pôr palha}
 
\subsection{Reconhecimento de Objectos Complexos}

Para focar os esforços de desnevolvimento definiu-se as \emph{mesas} como caso de estudo dos objectos
complexos. Esta escolha apoia-se no facto destes objectos relativamente simples terem
propriedades interessantes para o campo da robótica autónoma.

Um desses factores é o facto de ser um objecto bastante comum que pode ser encontrado em qualquer ambiente.

Outro factor é que, sendo as mesas objectos relativamente simples, a sua morfologia pode ser muito diferente de caso para caso.
Podemos ter mesas com um só apoio central, outras poderão ter três ou quatro apoios, e estes apoios podem
ter a forma de cilindros, paralelipípedos ou mesmo até formas menos regulares. Os tampos da mesa podem
também ter formas bastante diferentes: desde tampos rectângulares, circulares, ovais ou mesmo de formas 
menos regulares à semelhança dos seus apoios.

Existirá também um dicionário de mesas conhecidas, sobre as quais será produzido um factor de confiança de
as ter detectado que é calculado a partir do factor de confiança de se ter encontrado os seus constituintes.

%falar do facto de a omissão de dados ser muito menos grave do que a obtenção de dados errados.


\section{Resumo e Conclusões}

Esta dissertação fará então o reconhecimento de mesas, através do reconhecimentos dos seus constituíntes mais
simples, i.e.: planos horizontais com várias formas e cilindros e paralelipípedos e indicará a certeza que tem
de que é realmente uma mesa.