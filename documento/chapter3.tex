\chapter{Descrição do Problema}\label{chap:chap3}

\section*{}

%Este capítulo deve começar por fazer uma apresentação detalhada do problema a resolver\footnote{Na introdução a apresentação do   problema foi breve.} podendo mesmo, caso se justifique, constituir-se um capítulo com essa finalidade.

% Deve depois dedicar-se à apresentação da solução sem detalhes de implementação. Dependendo do trabalho, pode ser uma descrição mais teórica, mais 'arquitectural', etc.

\section{Reconhecimento de Objetos}


O objetivo desta tese é o reconhecimento de objetos através das perceções de um robô, tirando o máximo partido dos sensores 3D que recentemente apareceram no mercado. Utilizando tecnologia já desenvolvida, como os sensores 3D e bibliotecas informáticas, será desenvolvida uma aplicação onde se faça o reconhecimento de objetos simples, nomeadamente planos, paralelepípedo e cilindros, para se utilizar no reconhecimento de objetos mais complexos dos quais estes fazem parte.

Visto que o objetivo último é a integração no software de um demonstrador de robótica autónoma, um objetivo secundário é que o reconhecimento seja feito num curto espaço de tempo para maximizar a performance do demonstrador.

O trabalho desenvolvido também poderá ser utilizado, no demonstrador autónomo, para resolver o problema de localização, isto é: como é que um robô autónomo sabe a sua localização no espaço. Havendo reconhecimento de objetos permitirá que o robô utilize a posição dos objetos reconhecidos para mais facilmente se localizar no espaço.

%\section{ Solução do Problema}

Para resolver o problema do reconhecimento de objetos em 3D decidiu-se seguir uma estratégia \emph{bottom-up} tratando primeiro do reconhecimento de objetos simples, nomeadamente planos no espaço, cilindros e paralelepípedo, e de seguida reconhecer mesas em cuja constituição se encontram esses objetos. A metodologia será de seguir por uma fusão sensorial de modo a combinar toda a informação disponível para culminar numa avaliação precisa dos objetos nas imagens RGBD.

O sensor utilizado para a captura de informação 3D foi o \emph{Kinect} da \emph{Microsoft} que é o mais comum no mercado e também foi o que a comunidade \emph{opensource} mais rapidamente adotou, desenvolvendo controladores e bibliotecas que permitem que o desenvolvimento seja multi-plataforma.


\subsection{Reconhecimento de Objetos Simples}

No que diz respeito a objetos simples optou-se pelos seguintes que são os constituintes mais comuns
do principal caso de estudo dos objetos complexos:
\begin{itemize}
\item Cilindros;
\item Paralelepípedos;
\item Planos;
\end{itemize}

O reconhecimento destes objetos mais simples terá associado um valor de confiança para se poder avaliar a qualidade do método, contudo é preciso ter em atenção o custo em termos de tempo de processamento e comparar as vantagens que estes reconhecimentos intermédios trazem.
 
\subsection{Reconhecimento de Objetos Complexos}

Para focar os esforços de desenvolvimento definiu-se as \emph{mesas} como prova de conceito para os objetos complexos. Esta escolha apoia-se no facto destes objetos relativamente simples terem propriedades interessantes para o campo da robótica autónoma.

Um desses fatores é o facto de ser um objeto bastante comum que pode ser encontrado em qualquer ambiente.

Outro facto é que, sendo as mesas objetos relativamente simples, a sua morfologia pode ser muito diferente de caso para caso. Podemos ter mesas com um só apoio central, outras poderão ter três ou quatro apoios, e estes apoios podem ter a forma de cilindros, paralelepípedos ou mesmo até formas menos regulares. Os tampos da mesa podem também ter formas bastante diferentes: desde tampos retangulares, circulares, ovais ou mesmo de formas menos regulares à semelhança dos seus apoios.

Existirá também um dicionário de mesas conhecidas, sobre as quais será produzido um fator de confiança de as ter detetado que é calculado a partir do fator de confiança de se ter encontrado os seus constituintes.

%falar do facto de a omissão de dados ser muito menos grave do que a obtenção de dados errados.


\section{Resumo e Conclusões}

Esta dissertação fará então o reconhecimento de mesas, através do reconhecimentos dos seus constituintes mais simples, i.e.: planos horizontais com várias formas e cilindros e paralelepípedos e indicará a certeza que tem
de que é realmente uma mesa.