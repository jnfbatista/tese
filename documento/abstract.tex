\chapter{Resumo}

O aparecimento de dispositivos RGBD, ou seja, câmaras que além da imagem RGB
normal também captam informação de profundidade, disponíveis para o utilizador
comum mudaram a robótica a nível académico porque tornou fácilmente disponível
a qualquer investigador um sensor que de outra forma seria demasiado caro
para ser considerado. Desta forma, pode-se equipar um robô de demonstrações de
robótica autónoma com um \emph{Kinect} e usufruir de um sensor de profundidade.

Tendo disponível este tipo de sensor, torna-se necessário desenvolver 
software de reconhecimento em tempo real de objectos, tais como esferas, cones
e cilindros para poderem ser utilizados nas demonstrações de robótica
autónoma. Fazendo o robô reagir a objectos, tal como perseguir uma esfera ou
fugir de um dado cilindro, permite criar demonstrações interactivas e apelativas
para sensibilizar quem assiste ao mundo da robótica e do que através dela se pode
criar.

Todo este trabalho implica uma pesquisa científica cuidada dos métodos e técnicas
que existem para o reconhecimento dos ditos objectos e também dos de diferentes tipos
de demonstradores que existem que servem diferentes propósitos mas que demonstram
um exemplo de robótica autónoma.

Encontra-se também neste documento um planeamento das tarefas que serão realizadas
para a conclusão com sucesso desta dissertação.

