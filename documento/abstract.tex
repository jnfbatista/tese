\chapter*{Resumo}

O aparecimento de dispositivos RGBD, ou seja, sensores que além de captarem imagens RGB,
(como qualquer câmara) também captam informação de profundidade, disponíveis para o utilizador
comum tornou acessível um sensor que de outra forma seria demasiado caro
para um demonstrador de robótica autónoma de baixo custo. Desta forma, pode-se equipar um robô de demonstrações de robótica autónoma com o \emph{Kinect} e usufruir de um sensor de profundidade.

Fazendo o robô reagir a objetos simples, tal como perseguir uma esfera ou
fugir de um cilindro e mesmo a combinação de ambos os comportamentos, permite 
criar demonstrações interativas e apelativas para sensibilizar a assistência 
ao mundo da robótica e do que através dela se pode criar. Para concretizar estes objetivos,
tirando partido da disponibilidade deste tipo de sensor, torna-se necessário desenvolver 
software de reconhecimento em tempo real de objetos simples (esferas, cones
e cilindros) para poderem ser utilizados nas demonstrações de robótica
autónoma. 

Todo este trabalho implica uma pesquisa científica cuidada dos métodos e técnicas
que existem para o reconhecimento dos ditos objetos e também dos diferentes tipos
de demonstradores que existem, que embora sirvam diferentes propósitos, demonstram
exemplos de robótica autónoma.

É feito também um estudo da precisão do \emph{Kinect} na leitura que faz do posicionamento de objetos no espaço por ele detetado, sendo que se conclui que tem uma precisão bastante satisfatória.

Neste trabalho confirma-se que a possibilidade de identificação de objetos complexos é uma realidade, podendo ser feita com recurso à ajuda de bibliotecas informáticas e algumas metodologias simples, das quais se pode extrair resultados bastantes interessantes, sendo que foi utilizada a mesa como prova de conceito de objetos complexos.




\chapter*{Abstract}


The appearance of RGBD devices, which means that these are devices that not only do they capture the RGB information they also capture the depth of a pixel, and their availability as a consumer of the shelf device has made possible their incorporation in any robotic autonomous driving demonstrator cheaply. This way a \emph{Kinect} can be taken advantage of as a depth sensor.

Making the robot interact with simple objects, like follow a sphere, or avoid a cylinder or even combining both behaviours , allows the creation of a appealing and interactive demonstrations to sensitize the public to the world of robotics. To make this a reality, taking advantage of these kind of sensors, it becomes necessary to develop software to recognize simple objects in real time in order for them to be used in autonomous robotics demonstrations.

All this work implies a careful scientific review of the existing methods to recognize the said objects and also review the different kinds of demonstrators and their purpose.

It is also necessary to perform a evaluation of the \emph{Kinect}'s precision on evaluating objects placement in the space it detects, and it was found that its precision is very good.

This work confirms the ability to identify complex objects, and it can be done using software libraries and some simple methodologies, from which we can gather rather interesting results using the table as proof of concept to complex objects.
