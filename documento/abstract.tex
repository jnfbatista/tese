\chapter{Resumo}

O aparecimento de dispositivos RGBD, ou seja, sensores que além de captarem imagens RGB,
(como qualquer câmara) também captam informação de profundidade, disponíveis para o utilizador
comum tornou acessível um sensor que de outra forma seria demasiado caro
para um demonstrador de robótica autónoma de baixo custo. Desta forma, pode-se equipar um robô de demonstrações de robótica autónoma com um \emph{Kinect} e usufruir de um sensor de profundidade.

Fazendo o robô reagir a objetos simples, tal como perseguir uma esfera ou
fugir de um cilindro e mesmo a combinação de ambos os comportamentos, permite 
criar demonstrações interativas e apelativas para sensibilizar a assistência 
ao mundo da robótica e do que através dela se pode criar. Para concretizar estes objetivos,
tirando partido da disponibilidade deste tipo de sensor, torna-se necessário desenvolver 
software de reconhecimento em tempo real de objetos simples (esferas, cones
e cilindros) para poderem ser utilizados nas demonstrações de robótica
autónoma. 

Todo este trabalho implica uma pesquisa científica cuidada dos métodos e técnicas
que existem para o reconhecimento dos ditos objetos e também dos diferentes tipos
de demonstradores que existem, que embora sirvam diferentes propósitos, demonstram
exemplos de robótica autónoma.

É feito também um estudo da precisão do \emph{Kinect} na leitura que faz do posicionamento de objetos no espaço por ele detetado, sendo que se conclui que tem uma precisão bastante satisfatória.

Neste trabalho confirma-se que a possibilidade de identificação de objetos complexos é uma realidade, podendo ser feita com recurso à ajuda de bibliotecas informáticas e algumas metodologias simples, das quais se pode extrair resultados bastantes interessantes, sendo que foi utilizada as mesas como prova de conceito de objetos complexos.




