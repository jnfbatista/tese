\chapter{Implementação}\label{chap:chap4}

\section*{}

% Este capítulo pode ser dedicado à apresentação de detalhes de nível
% mais baixo relacionados com o enquadramento e implementação das
% soluções preconizadas no capítulo anterior.
% Note-se no entanto que detalhes desnecessários à compreensão do
% trabalho devem ser remetidos para anexos.

% Dependendo do volume, a avaliação do trabalho pode ser incluída neste
% capítulo ou pode constituir um capítulo separado.

O programa desenvolvido foi escrito em \emph{C++} recorrendo a algumas biliotecas 
informáticas, nomeadamente a \emph{Point Cloud Library} para a captura e manipulação
de nuvens de pontos, \emph{Boost}, \emph{Qt} para o desenvolvimento da interface
gráfica. De seguida são analisados com mais detalhe 

\section{Captura de Imagens 3D para análise }

A captura de imagens em 3D foi feita utilizando o \emph{Kinect} e a bilbioteca
\emph{Point Cloud Library}.

\section{ Análise selecção e Separação de amostra de controlo}

Seleccionou-se e separou-se algumas imagens com mesas para testar a qualidade
de detecção após o desenvolvimento da tese.

\section {Implementação da Detecção}

Primeiro é feito um pré processamente para remover o chão da imagem e um corte a X
metros de modo a eliminar algum ruído persistente na imagem e eliminar a zona de
onde o \emph{Kinect} perde precisão.

Depois é feita uma clusterização \footnote{isot vai ser passado a grupo de pontos}
usando o algoritmo KdTree para separar a imagem em
objectos coerentes e separáveis para não acontecer o caso de estarmos a reconhecer
o que parece uma perna de uma mesa a dois metros de distância do que é reconhecido
como um tampo.

De seguida cada um dos objectos separáveis passa por um processo onde se tenta 
detectar constituintes das mesas, como pernas e tampos. Caso sejam encontrados 
verifica-se se correspondem a alguma das mesas conhecidas que estão guardadas
na base de dados das mesas conhecidas\footnote{explicar como isto funciona}.
Com base no que foi detectado, utilizando uma metodologia fuzzy, foram combinados
os factores de confiança na detecção dos objectoss simples para dar um factor de 
confiança de ser alguma das mesas já conhecidas.


\section {Resultados Obtidos}

TBD

\section{Resumo}


