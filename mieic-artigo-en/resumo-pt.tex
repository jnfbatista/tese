%-----------------------------------------------
% Template para criação de resumos de projectos/dissertação
% jlopes AT fe.up.pt,   Fri Jul  3 11:08:59 2009
%-----------------------------------------------

\documentclass[9pt,a4paper]{extarticle}

%% English version: comment first, uncomment second
%\usepackage[portuguese]{babel}  % Portuguese
\usepackage[english]{babel}     % English
\usepackage{graphicx}           % images .png or .pdf w/ pdflatex OR .eps w/ latex
\usepackage{times}              % use Times type-1 fonts
\usepackage[utf8]{inputenc}     % 8 bits using UTF-8
\usepackage{url}                % URLs
\usepackage{multicol}           % twocolumn, etc
\usepackage{float}              % improve figures & tables floating
\usepackage[tableposition=top]{caption} % captions
%% English version: comment first (maybe)
%\usepackage{indentfirst}        % portuguese standard for paragraphs
\usepackage{parskip}

%% page layout
\usepackage[a4paper,margin=30mm,noheadfoot]{geometry}

%% space between columns
\columnsep 12mm

%% headers & footers
\pagestyle{empty}

%% figure & table caption
\captionsetup{figurename=Fig.,tablename=Tab.,labelsep=endash,font=bf,skip=.5\baselineskip}

%% heading
\makeatletter
\renewcommand*{\@seccntformat}[1]{%
  \csname the#1\endcsname.\quad
}
\makeatother

%% avoid widows and orphans
\clubpenalty=300
\widowpenalty=300

\begin{document}

\title{\vspace*{-8mm}\textbf{\textsc{Sistema de Reconhecimento de Objetos\\
para Demonstrador de Condução Robótica Autónoma}}}
\author{\emph{João Nuno Ferreira Batista}\\[2mm]
\small{Thesis supervised by \emph{Prof.\ Armando Jorge Sousa}}}
\date{}
\maketitle
%no page number 
\thispagestyle{empty}

\vspace*{-4mm}\noindent\rule{\textwidth}{0.4pt}\vspace*{4mm}

\begin{multicols}{2}

\section{Motivation}\label{sec:motiva}

The main motivation of this thesis is to do recognition of objects in real time using the
\emph{Kinect} sensor, and after being recognized by the system they are to be used to do autonomous
robotics demonstrations. The most important characteristics are: the recognition of 3D objects in
real time and the identifications of their characteristics.

\section{Goals}\label{sec:goals}

The objective of this thesis is the recognition of objects through the robots perceptions, making
full use of 3D sensors like the \emph{Kinect}. Using already developed technology, like 3D sensors
and e informatics libraries, an application will be developed where the recognition of simple objects
is made, in order to use it to detect more complex objects composed by the simple ones.

Tables were chosen as the complex object because of its morphological characteristics.

In sum, it is supposed to allow the following:
\begin{itemize}
\item Separate different objects on a RGBD image;
\item In each object detect if it is a table;
\item Compare its morphology with the existing tables in a dictionary;
\item Output a confidence value of how likely it is to be each of the tables in the dictionary.
\end{itemize}

\section{Description}\label{sec:work}

The work consists of using the tools made available by the  \emph{Point Cloud Library} to do a software
application that can identify tables in RGBD images with its characteristics described in a existing
dictionary.

The dictionary is a XML file that has all the tables' morphology.

The process, in short, is the following: the superfluous information in the RGBD image is removed, then
the objects are separated and, for each, its morphology is extracted and compared to the models stored
in the dictionary.

In the end, the result is if one of these objects is a table and what is the confidence value of being
each of the tables in the dictionary.


\section{Conclusions}\label{sec:conclui}

In this thesis, it was concluded that with simple and fast algorithms it is possible recognize tables in
RGBD images, but the process of separating diferent objects in an RGBD image is time consuming and with
big margin for improvement. Some fragilities of this methodology are exposed, like the inability to 
recognize tables in profile. Furthermore, it was possible to verify that the \emph{Kinect} has a very
acceptable precision.

%%English version: comment first, uncomment second
%\bibliographystyle{unsrt-pt}  % numeric, unsorted refs
%\bibliographystyle{unsrt}  % numeric, unsorted refs
%\bibliography{refs}

\end{multicols}

\end{document}
