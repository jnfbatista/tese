%-----------------------------------------------
% Template para criação de resumos de projectos/dissertação
% jlopes AT fe.up.pt,   Fri Jul  3 11:08:59 2009
%-----------------------------------------------

\documentclass[9pt,a4paper]{extarticle}

%% English version: comment first, uncomment second
\usepackage[portuguese]{babel}  % Portuguese
%\usepackage[english]{babel}     % English
\usepackage{graphicx}           % images .png or .pdf w/ pdflatex OR .eps w/ latex
\usepackage{times}              % use Times type-1 fonts
\usepackage[utf8]{inputenc}     % 8 bits using UTF-8
\usepackage{url}                % URLs
\usepackage{multicol}           % twocolumn, etc
\usepackage{float}              % improve figures & tables floating
\usepackage[tableposition=top]{caption} % captions
%% English version: comment first (maybe)
\usepackage{indentfirst}        % portuguese standard for paragraphs
%\usepackage{parskip}

%% page layout
\usepackage[a4paper,margin=30mm,noheadfoot]{geometry}

%% space between columns
\columnsep 12mm

%% headers & footers
\pagestyle{empty}

%% figure & table caption
\captionsetup{figurename=Fig.,tablename=Tab.,labelsep=endash,font=bf,skip=.5\baselineskip}

%% heading
\makeatletter
\renewcommand*{\@seccntformat}[1]{%
  \csname the#1\endcsname.\quad
}
\makeatother

%% avoid widows and orphans
\clubpenalty=300
\widowpenalty=300

\begin{document}

\title{\vspace*{-8mm}\textbf{\textsc{Sistema de Reconhecimento de Objetos\\
para Demonstrador de Condução Robótica Autónoma}}}
\author{\emph{João Nuno Ferreira Batista}\\[2mm]
\small{Projecto/Dissertação realizado sob a orientação do \emph{Prof.\ Armando Jorge Sousa}}}
\date{}
\maketitle
%no page number 
\thispagestyle{empty}

\vspace*{-4mm}\noindent\rule{\textwidth}{0.4pt}\vspace*{4mm}

\begin{multicols}{2}

\section{Motivação}\label{sec:motiva}

A motivação principal desta dissertação é fazer o reconhecimento de objetos em tempo real através do sensor do Kinect, e após reconhecidos pelo sistema, serem utilizados para fazer demonstrações de robótica autónoma. As características mais importantes serão: o reconhecimento de objetos 3D em tempo real e a identificação das características dos mesmos.

\section{Objectivos}\label{sec:goals}

O objetivo desta tese é o reconhecimento de objetos através das perceções de um robô, tirando o máximo partido dos sensores 3D, como o Kinect. Utilizando tecnologia já desenvolvida, como os sensores 3D e bibliotecas informáticas, será desenvolvida uma aplicação onde se faça o reconhecimento de objetos simples, para se utilizar no reconhecimento de objetos mais complexos dos quais estes fazem parte.

Decidiu-se pelas mesas como o objeto complexo para caso de estudo por causa suas características morfológicas.

Em suma, pretende-se:
\begin{itemize}
\item Separar objetos diferentes numa imagem RGBD; 
\item Em cada objeto identificar se é uma mesa;
\item Comparar a sua morfologia com mesas existentes num dicionário;
\item Fornecer um valor da confiança de ser cada uma das mesas do dicionário.
\end{itemize}

\section{Descrição do Trabalho}\label{sec:work}

O trabalho consiste em utilizar as ferramentas fornecidas pela biblioteca informática \emph{Point Cloud Library} para fazer um \emph{software} que consiga identificar mesas pré existentes num dicionário numa imagem RGBD.

O dicionário não é mais do que um ficheiro XML onde se descreve a morfologia das mesas.

O processo, de uma forma resumida, é o seguinte: remove-se de uma imagem RGBD tudo que não é de interesse analisar, de seguida separam-se os objetos restantes e, para cada um deles, extrai-se a sua morfologia e finalmente compara-se com os modelos guardados no dicionário.

No final o resultado obtido é se um desses objetos é uma mesa e qual o valor da confiança de ser cada uma das mesas do dicionário.


\section{Conclusões}\label{sec:conclui}

Nesta dissertação conclui-se com com algoritmos simples e rápidos é possível fazer o reconhecimento de mesas em imagens RGBD, contudo a separação de objetos na imagem é um processo demorado e com larga margem para melhoria. Expõem-se algumas fragilidades desta metodologia,  como não conseguir reconhecer mesas de perfil.

Além destas conclusões, foi possível verificar que a precisão do \emph{Kinect} é bastante aceitável considerando-se as medidas por ele fornecidos.

%%English version: comment first, uncomment second
%\bibliographystyle{unsrt-pt}  % numeric, unsorted refs
%\bibliographystyle{unsrt}  % numeric, unsorted refs
%\bibliography{refs}

\end{multicols}

\end{document}
